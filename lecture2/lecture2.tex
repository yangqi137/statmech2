% !TEX engine = xelatex

\documentclass[12pt,aps,pra,notitlepage]{revtex4-1}
\usepackage{amsmath,amssymb}
\usepackage{xeCJK}
\setCJKmainfont{STSong}
\begin{document}

\title{第二周:正则系综和巨正则系综}
\maketitle

\section{正则系综}
\label{sec:canonical}

上一节介绍的微正则系综适用于研究一个已知能量$E$的系统。
我们一般希望研究一个已知温度$T$,而不是一个已知能量的系统,因为一个热力学系统的能量很难直接测量。
为此,我们引入正则系综。
正则系综的温度恒定,而其能量则在涨落。
为了保持它的温度恒定,我们假设该系统和一个无穷大的温度为$T$的热库保持接触。
我们假设该热库(的热容)为无穷大,从而使得它在和系统发生能量交换的同时温度仍保持不变。
系统1与热库2一起构成一个能量守恒的孤立系统,因而可以用一个微正则系综描述:
\[\rho = \text{Const}.\]
这个微正则系综总能量为系统与热库能量之和:
\[E = E_1 + E_2.\]
对于系统$A_1$的某一个能量为$E(x)$的状态$x$,
\[\rho(x) = \frac1{\Omega(E)}\Omega^\prime(E-E(x)).\]

\section{正则系综和微正则系综的比较}

\section{巨正则系综}
\label{sec:grand}

\section{例子:无相互作用经典气体}
\section{例子:有相互作用经典气体--集团展开}
\end{document}
