% !TEX engine = xelatex

\documentclass[12pt,aps,pra,notitlepage]{revtex4-1}
\usepackage{amsmath,amssymb}
\usepackage{xeCJK}
\setCJKmainfont{STSong}
\DeclareMathOperator{\tr}{tr}
\begin{document}

\title{第八次作业}
\maketitle
\begin{center}
  11月19号截止。
\end{center}
\begin{enumerate}
  \item $O(N)$模型的Wolff更新算法:U. Wolff, Phys. Rev. Lett. \textbf{62}, 361 (1989).\\
  考虑一个$O(N)$模型:
  \[Z=\int\prod_i\mathcal{D}\vec{\sigma}_i\delta(\vec\sigma_i^2-1)\exp\left[\beta\sum_{\langle ij\rangle}\vec\sigma_i\cdot\vec\sigma_j\right].\]
  考虑如下的Wolff算法:
  \begin{enumerate}
    \item 随机选择一个初始点$i_0$。
    \item 随机选择一个$N$维平面,用它的法向单位矢量$\vec r$表示。我们可以定义相对这个平面的镜像反射操作$\vec\sigma\rightarrow R(\vec r)\vec\sigma$:
    \[R(\vec r)\vec\sigma = \vec\sigma - 2(\vec\sigma\cdot\vec r)\vec r.\]
    用这个操作把$\vec\sigma_i$变成$R(\vec r)\vec\sigma_i$。
    \item 我们以$i_0$作为初始点开始构造一个集团:从一个集团里的点$i$出发,把它的一个近邻$j$加入集团的概率为
    \[P(\vec\sigma_i,\vec\sigma_j) = 1-\exp\left\{\min\left[0, \beta\vec\sigma_i\cdot\vec\sigma_j-\beta\vec\sigma_i\cdot R(\vec r)\vec\sigma_j\right]\right\}.\]
    如果$j$被加入集团,把$\vec\sigma_j$变成$R(\vec r)\vec\sigma_j$。
    \item 重复上述过程,直到完成构造一个集团。
  \end{enumerate}
  证明上述算法满足细致平衡原理。

  \item 利用课上介绍的转移矩阵实空间重整化的做法,解决教材上习题14.1。
\end{enumerate}
\end{document}
