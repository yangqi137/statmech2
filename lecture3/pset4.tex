% !TEX engine = xelatex

\documentclass[12pt,aps,pra,notitlepage]{revtex4-1}
\usepackage{amsmath,amssymb}
\usepackage{xeCJK}
\setCJKmainfont{STSong}
\begin{document}

\title{第四次作业}
\maketitle
\begin{center}
  10月22号截止。
\end{center}
\begin{enumerate}
  \item Sachdev's Lecture Note, Problem 3.6\\
  说明:这里$N$分量的标量场指的是如下模型:
  \[Z=\int\prod_{\alpha}\mathcal Dy_{\alpha}
  \exp\left\{-\frac12\sum_{ij\alpha}\phi_{i\alpha}A_{ij}\phi_{j\alpha}
  -\frac u{4}\sum_i\sum_{\alpha\beta}\phi_{i\alpha}^2\phi_{i\beta}^2
  \right\}.\]
  其中相互作用系数习惯上写成$\frac u4$而不是$\frac u{4!}$,因为后面的项在$N$分量的时候只有$2!\times2!=4$重对称性。
  \item $\phi^3$场论。

  我们考虑如下的场论,
  \[Z(g)=\int\mathcal D\phi \exp\left(
  -\frac12\sum_{ij}\phi_iA_ij\phi_j-\frac g{3!}\sum_i\phi_i^3\right).\]
  用微扰论及费曼图计算如下物理量:
  \begin{enumerate}
    \item 计算$\langle\phi_i\rangle$最低阶非零贡献的结果。为什么在这个模型中$\langle\phi_i\rangle\neq0$,而在$\phi^4$模型中$\langle\phi_i\rangle=0$?
    \item 计算$\langle\phi_i\phi_j\rangle$到$g^2$阶的结果。注意$\langle\phi_i\rangle\neq0$。
  \end{enumerate}
  \item PB 16.9\\
  提示:注意Finite Size Effect:数值模拟中我们研究的是有限大小体系,而理论计算中得到的是热力学极限($L\rightarrow\infty$)下的结果。如何比较这两个结果?建议在数值模拟中选取$L=10\sim1000$。
\end{enumerate}
\end{document}
